\phantomsection
\chapter*{毕业设计小结}
\addcontentsline{toc}{chapter}{\fHei 毕业设计小结}

毕业论文是大学四年的最后一份大作业...


可听化(Auralization)[1]是近年来随着声学仿真技术的长足发展而出现的新概念,它的具体含义是通过对一包含单个(或者多个)声源的声场进行物理或数学建模,以达到声音绘制(Audio rendering)或称声学仿真(Acoustical simulation)的目的。这样,人们可以获得该声场中任意位置的双耳听觉感受。换句话说,可听化技术在客观上主要是模拟特定声场(包括声源、声传播环境以及聆听者三要素)中声音传播的物理过程,从而使其中的聆听者作为一个主体能够获得对整个场景声学特性的主观感知[2~5]。


\clearpage
\endinput